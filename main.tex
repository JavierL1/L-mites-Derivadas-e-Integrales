%%%%%%%%%%%%%%%%%%%%%%%%%%%%%%%%%%%%%%%%%
% Tufte-Style Book (Documentation Template)
% LaTeX Template
% Version 1.0 (5/1/13)
%
% This template has been downloaded from:
% http://www.LaTeXTemplates.com
%
% Original author:
% The Tufte-LaTeX Developers (tufte-latex.googlecode.com) 
%
% License:
% Apache License (Version 2.0)
%
% IMPORTANT NOTE:
% In addition to running BibTeX to compile the reference list from the .bib
% file, you will need to run MakeIndex to compile the index at the end of the
% document.
%
%%%%%%%%%%%%%%%%%%%%%%%%%%%%%%%%%%%%%%%%% 
%----------------------------------------------------------------------------------------
%	PACKAGES AND OTHER DOCUMENT CONFIGURATIONS
%----------------------------------------------------------------------------------------

\documentclass[justified,twoside,symmetric,explicit,letterpaper]{tufte-book} % Use the tufte-book class which in turn uses the tufte-common class
\geometry{
left=3cm
}
\usepackage[most]{tcolorbox}
\usepackage{wallpaper}
\usepackage[utf8]{inputenc}
\usepackage[spanish]{babel}
%\hypersetup{colorlinks} % Comment this line if you don't wish to have colored links
\usepackage{eurosym} %Euro symbolf
\usepackage{microtype} % Improves character and word spacing
\usepackage{wrapfig} %Wrap a figure in text. Used in margin boxes.
\usepackage{float}
\usepackage{tikz}
\usetikzlibrary{shapes,positioning}
\usepackage[export]{adjustbox}
\usepackage{tocloft} % Table of contents customization
\usepackage{lipsum} % Inserts dummy text
\usepackage{lmodern}
\usepackage{enumerate}
\usepackage{booktabs} % Better horizontal rules in tables
\usepackage{subcaption} %For the subfigure environment
\captionsetup{compatibility=false}

\usepackage{array} % Table column width manipulation
\usepackage[format=plain, font={footnotesize}]{caption}
\setlength{\abovecaptionskip}{2pt plus 3pt minus 2pt}
\usepackage{amsmath} %For math formatting
\numberwithin{figure}{chapter}

\usepackage{amssymb} %For blacksquares

\usepackage{graphicx} % Needed to insert images into the document
\graphicspath{{./ImagesMain/}} % Sets the default location of pictures
\setkeys{Gin}{width=\linewidth,totalheight=\textheight,keepaspectratio} % Improves figure scaling

\usepackage{fancyvrb} % Allows customization of verbatim environments
\fvset{fontsize=\normalsize} % The font size of all verbatim text can be changed here
\newcommand{\hangp}[1]{\makebox[0pt][r]{(}#1\makebox[0pt][l]{)}} % New command to create parentheses around text in tables which take up no horizontal space - this improves column spacing
\newcommand{\hangstar}{\makebox[0pt][l]{*}} % New command to create asterisks in tables which take up no horizontal space - this improves column spacing
\usepackage{enumitem}
%\setlist[enumerate]{wide=\parindent,topsep=0cm}
\setlist[enumerate]{topsep=0cm}
\setlist[enumerate,2]{label=\alph*), wide=2\parindent}
\usepackage{xspace} % Used for printing a trailing space better than using a tilde (~) using the \xspace command
\usepackage{hhline}
\usepackage{multirow}
\usepackage{wrapfig}
\usepackage{multicol}
\newcommand{\monthyear}{\ifcase\month\or January\or February\or March\or April\or May\or June\or July\or August\or September\or October\or November\or December\fi\space\number\year} % A command to print the current month and year
\usetikzlibrary{calc}
\newcommand{\openepigraph}[2]{ % This block sets up a command for printing an epigraph with 2 arguments - the quote and the author
\begin{fullwidth}
\sffamily\large
\begin{doublespace}
\noindent\allcaps{#1}\\ % The quote
\noindent\allcaps{#2} % The author
\end{doublespace}
\end{fullwidth}
}


\newcommand{\blankpage}{\newpage\hbox{}\thispagestyle{empty}\newpage} % Command to insert a blank page

\usepackage{units} % Used for printing standard units

\newcommand{\hairsp}{\hspace{1pt}} % Command to print a very short space
\newcommand{\ie}{\textit{i.\hairsp{}e.}\xspace} % Command to print i.e.
\newcommand{\eg}{\textit{e.\hairsp{}g.}\xspace} % Command to print e.g.
\newcommand{\na}{\quad--} % Used in tables for N/A cells
\newcommand{\measure}[3]{#1/#2$\times$\unit[#3]{pc}} % Typesets the font size, leading, and measure in the form of: 10/12x26 pc.
\newcommand{\tuftebs}{\symbol{'134}} % Command to print a backslash in tt type in OT1/T1


% This block contains a number of shortcuts used throughout the book
\newcommand{\vdqi}{\textit{VDQI}\xspace}
\newcommand{\ei}{\textit{EI}\xspace}
\newcommand{\ve}{\textit{VE}\xspace}
\newcommand{\be}{\textit{BE}\xspace}
\newcommand{\VDQI}{\textit{The Visual Display of Quantitative Information}\xspace}
\newcommand{\EI}{\textit{Envisioning Information}\xspace}
\newcommand{\VE}{\textit{Visual Explanations}\xspace}
\newcommand{\BE}{\textit{Beautiful Evidence}\xspace}
\newcommand{\TL}{Tufte-\LaTeX\xspace}

%---------------Cosas que yo defino--------------------------------%


\renewcommand{\arraystretch}{1.3} % Adding more space between rows in table

\newcolumntype{L}[1]{>{\raggedright\let\newline\\\arraybackslash\hspace{0pt}}m{#1}}

\newcolumntype{C}[1]{>{\centering\let\newline\\\arraybackslash\hspace{0pt}}m{#1}}

\newcolumntype{R}[1]{>{\raggedleft\let\newline\\\arraybackslash\hspace{0pt}}m{#1}}

\setcounter{tocdepth}{1}
\definecolor{gray75}{gray}{0.75}
\newcommand{\hsp}{\hspace{20pt}}

\setcounter{secnumdepth}{2}

\newcommand{\margindyk}[2][0pt]{
\marginnote[#1]{
\begin{tcolorbox}[title=Sabías que...?]
  \begin{wrapfigure}[3]{R}{0.1\textwidth}
  \centering
  \vspace*{-10pt}
  \includegraphics[width=0.7cm]{question1}
  \end{wrapfigure}
  
  #2
\end{tcolorbox}
}
}

\newcommand{\marginddi}[2][0pt]{
\marginnote[#1]{
\begin{tcolorbox}[title=Usted no lo haga]
  \begin{wrapfigure}[3]{R}{0.1\textwidth}
  \centering
  \vspace*{-10pt}
  \includegraphics[width=0.7cm]{biological}
  \end{wrapfigure}
  
  #2
\end{tcolorbox}
}
}

\newcommand{\marginobs}[2][0pt]{
\marginnote[#1]{
\begin{tcolorbox}[title=Observación]
  \begin{wrapfigure}[3]{R}{0.1\textwidth}
  \centering
  \vspace*{-10pt}
  \includegraphics[width=0.7cm]{lupa}
  \end{wrapfigure}
  
  #2
\end{tcolorbox}
}
}

\newenvironment{rayas}[1]{
    \begin{tcolorbox}[
    enhanced jigsaw, 
    breakable, 
    opacityback=0, 
    check odd page, 
    toggle left and right, 
    grow to right
    by=\marginparwidth+\marginparsep-5.7pt, 
    toggle enlargement=evenpage,
    left=0mm,
    right=0mm,
    frame hidden,
    overlay unbroken={%
            \draw[line width=0.2mm,black] ($(interior.north west)+(-0.25,0)$)--($(interior.north east)+(0.25,0)$);
            \draw[line width=0.2mm,black] ($(interior.south west)+(-0.25,0)$)--($(interior.south east)+(0.25,0)$);
        },
    overlay first={
            \draw[line width=0.2mm,black] ($(interior.north west)+(-0.25,0)$)--($(interior.north east)+(0.25,0)$);
        },
    overlay last={
            \draw[line width=0.2mm,black] ($(interior.south west)+(-0.25,0)$)--($(interior.south east)+(0.25,0)$);
        },
    ]
	\noindent \textbf{#1:}
	}{
    \end{tcolorbox}
    }
    
\newcommand{\bsqr}{
\hspace{0.4pt}
\scalebox{0.4}{$\blacksquare$}
\hspace{2pt}
}

\newtcolorbox{complec}{
    enhanced jigsaw, 
    breakable, 
    opacityback=0,
    check odd page, 
    toggle left and right, 
    grow to right by=\marginparwidth+\marginparsep-5.7pt, 
    toggle enlargement=evenpage,
    left=2mm,
    right=2mm,
    top=2mm,
    bottom=2mm,
    frame hidden,
    overlay unbroken={%
            
            \draw[line width=0.2mm,black] (title.north west)--(title.north east);
            \draw[line width=0.2mm,black] (interior.north west)--(interior.north east);
            \draw[line width=0.2mm,black] (interior.south west)--(interior.south east);
        },
    overlay first={
            \draw[line width=0.2mm,black] (title.north west)--(title.north east);
            \draw[line width=0.2mm,black] (interior.north west)--(interior.north east);
        },
    overlay last={
            \draw[line width=0.2mm,black] (interior.south west)--(interior.south east);
        },
    title=
    {\begin{center}
        \begin{minipage}{0.1\textwidth}
            \includegraphics[width=0.6cm]{comprensionlectora}
        \end{minipage}
        \begin{minipage}{0.5\textwidth}
            \centering
            \textsl{Comprensión lectora}
        \end{minipage}
        \begin{minipage}{0.1\textwidth}
            \centering
            \includegraphics[width=0.6cm]{comprensionlectora}
        \end{minipage}
    \end{center}},
    coltitle=black
}


\newcounter{nejerci}[chapter]

\newcommand{\nejer}{

    \stepcounter{nejerci}
    \item[\arabic{nejerci}.]
}

\newcounter{nfigure}[chapter]

\newcommand{\nfig}{
    \stepcounter{nfigure}
    Figura \thechapter.\arabic{nfigure}:
}


\tcbset{rightanswers/.style={
    breakable,
    enhanced,
    outer arc=0pt,
    arc=0pt,
    %frame hidden=false,
    colframe=black,
    colback=white,
    coltitle=black,
    check odd page, 
    toggle left and right, 
    grow to right by=\marginparwidth+\marginparsep-5.7pt, 
    toggle enlargement=evenpage,
    attach boxed title to top center={
    xshift=0mm, yshift=-0.4mm},
    boxed title style={
    colback=black!15,
    top=3pt,
    bottom=3pt,
    boxrule=0.5mm,
    sharp corners = south
    },
    fonttitle=\sffamily
  }
}

\newtcolorbox{respcorrectas}[1][]{
    rightanswers,
    title=Alternativas Correctas,
}

\newcommand\alter[2][]{\tikz[overlay]\node[fill=black!15,inner sep=2pt, anchor=text, circle,#1] {#2};\phantom{#2}}

% 'objetivo' Environment Definition
% Image Underlay Set on Top Right of the TCB
\tcbset{objective/.style={%
enhanced,
underlay={%
\begin{tcbclipinterior}
    \begin{scope}[opacity=0.2]
        \node[anchor=north]
         at ($(interior.north east)+(-1.2,0)$) {%
        \includegraphics[%
        width=1.5cm]{objective.png}};
    \end{scope}

\end{tcbclipinterior}
    }
}}
% Styling and Definirion of the Environment
\newtcolorbox{objetivo}{
    objective,
    enhanced,
    colback=white,
    colframe=black,
    coltext=black,
  	attach boxed title to top left={yshift=-3mm,yshifttext=-1mm},
    fontupper=\normalsize,
    arc=3mm,
    boxrule=0.5mm,
    boxsep=2mm, 	
    fonttitle=\large\bfseries,
    title=Objetivo PSU
}

% 'discusion' Environment Definition
\tcbset{discussion/.style={
    breakable,
    enhanced,
    frame hidden=true,
    colframe=black,
    colback=white,
    coltitle=black,
    check odd page, 
    left=0mm,
    right=0mm,
    toggle left and right, 
    toggle enlargement=evenpage,
    attach boxed title to top left={
    xshift=0mm},
    boxed title style={
    colback=black!10,
    boxrule=0.5mm,
    arc=4mm,
    boxsep=0mm
    }
  }
}


\newtcolorbox{discusion}[1][]{
    discussion,
    overlay unbroken={
        \draw[line width=0.2mm,black] (title.east)--($(title.east)+(5.98,0)$);
        \draw[line width=0.2mm,black] (frame.south west)--(frame.south east);
    },
    overlay first={
            \draw[line width=0.2mm,black] (title.east)--($(title.east)+(5.98,0)$);
        },
    overlay last={
            \draw[line width=0.2mm,black] (frame.south west)--(frame.south east);
    },
    title={
        \begin{minipage}{0.3\textwidth}
        \centering
           \textsl{\large\bfseries Para discutir}
        \end{minipage}
        \begin{minipage}{0.1\textwidth} \includegraphics[width=1cm]{paradiscutir}
        \end{minipage}
    },
}


% 'resumen' Environment Definition
\newtcolorbox{resumen}{%
enhanced,
breakable, 
attach boxed title to top center={yshift=-3mm,yshifttext=-1mm},
colback=black!15,
colframe=black,
colbacktitle=white,
title={\Large Resumen},
fonttitle=\bfseries,
coltitle=black,
boxed title style={size=title,colframe=black},
check odd page, 
toggle left and right, 
grow to right by=\marginparwidth+\marginparsep-5.7pt, 
toggle enlargement=evenpage,
}

% 'breakline' Environment Definition
\makeatletter
\newcommand{\breakline}{
\begin{tikzpicture}
\path[use as bounding box] (0,0) -- (\linewidth,0);
\draw[color=white!40!black,dashed,dash phase=2pt]
      (0-\kvtcb@leftlower-\kvtcb@boxsep,0)--
      (\linewidth+\kvtcb@rightlower+\kvtcb@boxsep,0);
\end{tikzpicture}
}
\makeatother

% 'subresumen' Environment Definition
\newtcolorbox{subresumen}[1][uwu]{%
enhanced,
breakable, 
colback=white,
colframe=black,
box align = top,
equal height group = {#1}
}

% 'evaluacion' Environment Definition
\tcbset{evaluation/.style={
    breakable,
    enhanced,
    outer arc=0pt,
    arc=0pt,
    frame hidden=true,
    colframe=black,
    colback=white,
    coltitle=black,
    attach boxed title to top left,
    check odd page, 
    toggle left and right, 
    grow to right by=\marginparwidth+\marginparsep-5.7pt, 
    toggle enlargement=evenpage,
    attach boxed title to top left={
    xshift=-0.25mm},
    boxed title style={
    colback=black!15,
    top=3pt,
    bottom=3pt,
    boxrule=0.5mm,
    sharp corners = west
    },
    fonttitle=\sffamily
  }
}

\newtcolorbox{evaluacion}[1][]{
  evaluation,
  title={\Large Evaluación de Unidad},
  overlay unbroken={
            \draw[line width=0.5mm,black] (interior.north west)--(interior.south west);
  },
  overlay first={
            \draw[line width=0.5mm,black] (frame.north west)--(frame.south west);
        },
        overlay middle={
        \draw[line width=0.5mm,black] (frame.north west)--(frame.south west);
        },
    overlay last={
            \draw[line width=0.5mm,black] (frame.north west)--(frame.south west);
            \draw[line width=0.5mm,black] (frame.south west)--(frame.south east);
        },
}

%----------------------------------------------------------------------------------------
%	Titulo
%----------------------------------------------------------------------------------------

\renewcommand{\maketitlepage}{%maketitle redefining for centering of the title
  \cleardoublepage
  {%
  \sffamily
  \begin{fullwidth}%
  \fontsize{18}{20}\selectfont\par\noindent\textcolor{darkgray}{\allcaps\thanklessauthor}%
  \vspace{11.5pc}%
  \begin{center}
  \fontsize{36}{40}\selectfont\par\noindent\textcolor{darkgray}{\allcaps\thanklesstitle}%
  \end{center}
  \vfill
  \fontsize{14}{16}\selectfont\par\noindent\allcaps\thanklesspublisher%
  \end{fullwidth}%
  }%
  \thispagestyle{empty}%
  \clearpage
}

\title[L\'imites, Derivadas e Integrales]{L\'imites, Derivadas e Integrales}

\author{Autor} % Author

\publisher{Autor} % Publisher

%----------------------------------------------------------------------------------------

\begin{document}

\frontmatter


%----------------------------------------------------------------------------------------





\maketitle % Print the title page

%----------------------------------------------------------------------------------------
%	COPYRIGHT PAGE
%----------------------------------------------------------------------------------------

\newpage
\begin{fullwidth}
~\vfill
\thispagestyle{empty}
\setlength{\parindent}{0pt}
\setlength{\parskip}{\baselineskip}

\begin{Large}L\'IMITES, DERIVADAS E INTEGRALES 2020\end{Large}\bigskip


\par\textbf{AUTORÍA}\medskip



\par\textbf{DISEÑO Y DIAGRAMACIÓN}\medskip



\par\textbf{EDICIÓN}\medskip





%\includegraphics[width=3cm]{cc}  \medskip

%\copyright \hspace{0.2cm} El presente documento está licenciado bajo la licencia Creative Commons Reconocimiento 3.0 Unported, bajo las siguientes condiciones:\smallskip 

%\noindent\textbf{Reconocimiento -- No Comercial -- Compartir Igual [by - nc - sa]:} No se permite el uso, comercial ni de la obra original ni de sus derivadas, la distribución de las cuales debe hacerse con la misma licencia que la obra original. 



\par\smallcaps{Publicado por \thanklesspublisher}

\par\textit{Primera edición, Enero 2020}

\end{fullwidth}





%----------------------------------------------------------------------------------------
%	PÁGINA DE DEDICACIÓN
%----------------------------------------------------------------------------------------

\cleardoublepage
~\vfill
\begin{fullwidth}
\begin{doublespace}
\noindent\fontsize{18}{22}\selectfont\itshape
\nohyphenation
\begin{flushright}
Dedicado
\end{flushright}
\end{doublespace}
\end{fullwidth}
\vfill
\vfill

%----------------------------------------------------------------------------------------
%	PAGINA DE PRESENTACION
%----------------------------------------------------------------------------------------

\clearpage


{\noindent\fontsize{30}{32}\selectfont Presentación \par}

\bigskip

\begin{fullwidth}
\noindent \textbf{En este  libro...}\medskip
\end{fullwidth}

\pagebreak

\begin{fullwidth}
%----------------------------------------------------------------------------------------------
%    Indice
%---------------------------------------------------------------------------------------------
\renewcommand{\cftchapfont}{Unidad \scshape}
\renewcommand\cftchapafterpnum{\par\addvspace{-1pt}}
\renewcommand\cftsecafterpnum{\par\addvspace{0pt}}
%\cleardoublepage
\tableofcontents % Print the table of contents
\addtocontents{toc}{~\hfill\textbf{Página}\par}
\end{fullwidth}
%--------------------------------------------------------------------------Chapter/Section/Subsection/Subsubsection Styles Definition ------------------------------------------------------%

%Chapter format
%% Command to hold chapter illustration image
\newcommand\chapterillustration{}
\newcommand\chapterwatermark{}

%% Define how the chapter title is printed
\titleformat{\chapter}{}{}{0pt}{
%% Background image at top of page
\ThisULCornerWallPaper{1}{\chapterillustration}
%% Draw a semi-transparent rectangle across the top
\tikz[overlay,remember picture]
  \fill[blue!80!white,opacity=.7]
  (current page.north west) rectangle 
  ([yshift=-3cm] current page.north east);
  %% "logo" image at lower right
  %% corner; Chapter number printed near spine
  %% edge (near the left); chapter title printed
  %% near outer edge (near the right).
  \tikz[remember picture,overlay] \node[opacity=0.25,inner sep=0pt,xshift=-30mm,yshift=35mm] at (current page.south east){\includegraphics[width=5cm]{\chapterwatermark}};
  \begin{tikzpicture}[overlay,remember picture]
  \node[anchor=south west,
  xshift=20mm,yshift=-30mm,
  font=\sffamily\bfseries\scshape\huge] 
  at (current page.north west) 
  {Unidad\ \thechapter};
  \node[fill=blue!80!black,text=white,
  font=\Huge\bfseries, 
  inner ysep=12pt, inner xsep=20pt,
  rounded rectangle,anchor=east, 
  xshift=-10mm,yshift=-30mm,align=right] 
  at (current page.north east) {#1};
  \end{tikzpicture}
}
\titlespacing*{\chapter}{0pt}{0pt}{135mm}

%Section format
\titleformat{\section}[hang]
{\huge\bfseries}
{\thesection\hsp\textcolor{gray75}{|}\hsp}
{0pt}
{\huge\bfseries #1}

%Subsection format
\titleformat{\subsection}
{\large}
{\thesubsection}
{5pt}
{\large\itshape#1}

%Subsubsection format
\renewcommand{\subsubsection}[1]{
    \medskip\noindent\textsl{#1}\medskip
}

%----------------------------------------------------------------------------------------
%	PRIMERA UNIDAD
%----------------------------------------------------------------------------------------
\mainmatter
\setcounter{chapter}{-1}

\renewcommand\chapterillustration{cover0}
\renewcommand\chapterwatermark{logo0}
\chapter{Preliminares}

\par En este cap\'itulo...


\begin{enumerate}[itemsep=2pt, leftmargin=10pt,itemindent=10pt]
    \item[\textbf{0.1}] 
    \item[\textbf{0.2}] 
\end{enumerate}


\label{ch:unidad0}
\clearpage

\graphicspath{{./ImagesUnit0/}}

\section{}



%----------------------------------------------------------------------------------------
%	SEGUNDA UNIDAD
%----------------------------------------------------------------------------------------
\mainmatter
\graphicspath{{./ImagesMain/}} 
\setcounter{chapter}{0}

\renewcommand\chapterillustration{cover1}
\renewcommand\chapterwatermark{logo1}
\chapter{L\'imites}

\par En este cap\'itulo...


\begin{enumerate}[itemsep=2pt, leftmargin=10pt,itemindent=10pt]
    \item[\textbf{1.1}] 
    \item[\textbf{1.2}] 
    \end{enumerate}


\label{ch:unidad1}
\clearpage

\graphicspath{{./ImagesUnit1/}}

\section{}


%----------------------------------------------------------------------------------------
%	SEGUNDA UNIDAD
%----------------------------------------------------------------------------------------
\mainmatter
\graphicspath{{./ImagesMain/}} 
\setcounter{chapter}{1}

\renewcommand\chapterillustration{cover2}
\renewcommand\chapterwatermark{logo2}
\chapter{Derivadas}

\par En este cap\'itulo...


\begin{enumerate}[itemsep=2pt, leftmargin=10pt,itemindent=10pt]
    \item[\textbf{2.1}] 
    \item[\textbf{2.2}] 
    \end{enumerate}



\label{ch:unidad2}
\clearpage

\graphicspath{{./ImagesUnit2/}}

\section{}

%----------------------------------------------------------------------------------------
%    TERCERA UNIDAD
%----------------------------------------------------------------------------------------
\mainmatter
\graphicspath{{./ImagesMain/}} 
\setcounter{chapter}{2}

\renewcommand\chapterillustration{cover3}
\renewcommand\chapterwatermark{logo3}

\chapter{Integrales}

\par En este cap\'itulo...


 \begin{enumerate}[itemsep=2pt, leftmargin=10pt,itemindent=10pt]
     \item[\textbf{3.1}] 
     \item[\textbf{3.2}] 

\end{enumerate}



\label{ch:unidad3}
\clearpage

\graphicspath{{./ImagesUnit3/}}

\section{}



\end{document}